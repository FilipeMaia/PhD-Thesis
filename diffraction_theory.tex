\chapter{Theory of X-ray Diffraction by Matter}\label{diffraction_theory}\noindent

This chapter gives an overview of the most important tools necessary to
understand X-ray diffraction and provides a simple framework for analysing most
CXDI experiments based on the first Born approximation, also known as
"kinematical" or "single-scattering" approximation. This will be of fundamental
importance when we later try to reconstruct the object that gave rise to a given
diffraction pattern, as  it is obviously impossible to do this if we cannot
predict the diffraction pattern that a given object produces.

\section{Diffraction by a free electron}\label{diffraction_physics}

According to classical electromagnetic theory when a plane monochromatic wave of
amplitude $E_0$, frequency $\nu$, propagating along the $z$ axis, given by
\begin{equation}
E_i(z,t) = E_0 \exp(2 \pi i \nu (t-z/c)) \, ,
\end{equation}
travels through an electron of charge $e$ and mass $m$ located at the origin of a coordinate
system, that electron will oscilate in the direction of the incident electric 
vector driven by
\begin{equation}
a(t) = \frac{e E_i}{m}
\end{equation}
with a frequency equal to the incoming wave. This in turn will make it radiate
an electric field $E_s$, like any accelerating charge. The electric field
generated by an accelerating electron is given by
\begin{equation}
E( \mathbf r,t) = \frac{e a_{\perp}(t - |\mathbf r|/c)}{4 \pi \epsilon_0 c^2 r}
\end{equation}
where  $\epsilon_0$ is the permittivity of free space and $a_{\perp}$ is the
acceleration projected on a plane normal to $r$, also called the transverse
component of the acceleration.

The radiated field will
have maximum strength on the direction of $E_i$ and zero strength perpedicular
to it
and emit an electric field, $E_s$, described by


using the born-approximation.

- Define Continuous Fourier Transform 

- State some handy properties of the fourier transform like like:

* Symmetry from real functions
* Friedel's law 
* Convolution theorem
* Autocorrelation

- Define DFT

- Talk about sampling and oversampling and define oversampled

- Nyquist frequency

- Aliasing

- Scattering by a free electron

- Scattering by an arbitrary electron density
