\chapter{Introduction}\label{introduction}\noindent

Mankind has always had a fascination with looking at the infinitely
small. Since the invention of microscope people have been trying to find ways to
look at smaller and smaller things. The invention of X-ray crystallography in
the beggining of the twentieth century made it possible to observe the
structures of things much smaller than the wavelength of visible light.

Nowadays X-ray crystallography is one of the most successful techniques ever
developed, and every year thousands of new protein structures are solved and deposited
in the Protein Data Bank. It is currently the only technique capable of
delivering atomic resolution images of biological macromolecules which are
invaluable in, for example, the understanding of biology or in the creation of
new drugs. Cryo electron microscopy also shows great promise with current
structures frequently achieving resolutions below 10 \AA, but still not enough
for atomic resolution.

However X-ray crystallography, as the name suggests, is limited to systems that
can be crystallized. Many, if not most, systems of biological interest are very
difficult, or impossible to crystallize. Probably the most striking example is
that of a simple cell, but there are many others such as organelles and many
membrane proteins. Alternative approaches have to be employed to image these
samples, including cryo electron microscopy, atomic force microscopy or
transmission electron microscopy. Unfortunately all of these alternatives have
serious drawbacks compared to X-ray crystallography such as not being able to
image the interior of the samples, providing a lower resolution or requiring
extensive sample preparation which might potential alter it.

Ultrafast Coherent X-ray Diffractive Imaging(CXDI) is a relatively new
technique, which uses a coherent, short and extremely
bright pulse of X-rays to capture a diffraction image of the sample which is
then phased to reveal the sample structure, has the potential to allow three
dimensional imaging of nonperiodic reproducible biological samples up to atomic
resolution and two dimensional imaging of non reproducible samples up to
very high resolution, without the need of slicing or staining the sample.

CXDI requires extremely a very high number of photons as it does not benefict from the
amplification effect of a crystal as X-ray crystallography does. But on the
other hand it is possible to sample the diffraction pattern in between the Bragg
peaks which makes the phasing problem much simpler. To be able to achieve high
resolution using CXDI very short pulses are necessary otherwise the radiation
damage that develops during the exposure limits the maximum resolution. For
example the highest resolution possible for biological samples using current
synchrotron based x-ray microscopes is around 20nm, limited by radiation damage.

The recent development of X-ray free electron lasers (XFELs) is the perfect
instrument to realize the full potential CXDI. XFELs can produce extremely
intense X-ray pulses, a billion times more brilliant than third
generation synchrotron sources, which are also extremely short, on the order of
only a couple femtoseconds. FLASH, in Hamburg, Germany, was the first soft X-ray
free electron laser in the world, and is based on the Self Amplified Spontaneous
Emission (SASE) principle. It started operations in the summer of 2005 at
a wavelength of 32nm and a peak power on the order of GW and pulse length up to
10fs. It has gone through several updates reaching the wavelength of
6.5nm. Despite being the test facility for the European X-ray Free Electron
Laser it has produced many important results in the CXDI field \cite{Cowboys, Chapman Holography}. In April 2009 the Linac Coherent Light Source (LCLS)
became the first hard X-ray free electron laser in the world producing light
with a wavelength of \mbox{1.5 \AA}. The Sprint-8 Compact SASE Source (SCSS) in
Japan will soon be ready and the European X-ray Free Electron Laser (XFEL)
will follow in a few years.

During the last few years there has also been a rapid development of tabletop
high harmonic generation (HHG) sources. These have the potential to one day
compete with free electron lasers and they are much more affordable for
individual labs. Nowadays there are HHG sources with very good coherence
properties capable of producing extremely short pulses, under 1fs, at soft X-ray wavelengths and more than $10^{11}$ photons
per pulse \cite{Ravasio}. This is still a few orders of magnitude below what is
possible using XFELs but due to the lower cost and greater access they might
become important tools in the near future.

The availability in the near future of several hard XFELs combined with the
increasing development of CXDI experimental techniques and data processing
algorithms have the potential to transform Ultrafast Coherent X-ray Diffractive
Imaging from a niche unconventional technique into a mainstream structural
biology tool that complements X-ray crystallography, as well as a general
imaging method such as electron microscopy.


%I'm pretty sure i need to reformulate this 
The aims of this thesis are: to present recent experimental results in Ultrafast
Coherent X-ray Diffractive Imaging using both free electron lasers and
HHG sources, to theoretically investigate the problem of sample heterogeneity
for reproducible samples and to propose new experiments made possible with this
new sources.
%%% Local Variables: 
%%% mode: latex
%%% TeX-master: "Thesis"
%%% End: 
