\chapter{Future Perspectives}\label{Future Perspectives}\noindent

LCLS has just started user operations. A revamped FLASH will come back to life
in August 2010. Fermi in Trieste will become operational in the autumn, and the
European XFEL will be ready in 2014. In the meantime, table-top and pocket-sized
instruments are popping up. These are very interesting times. The experimental
verification that it is possible to obtain a high resolution image of a sample
using ultrafast CXDI before the sample is turned into plasma ({\bf Paper
  \ref{cowboys}}) opens the door to many exciting possibilities in the field of
structural biology and nano imaging in general.

The possibility of obtaining high resolution 3D structures of reproducible
objects such as virus or proteins is within our horizon. Still formidable
problems have to be tackled, due to the extremely low signal to noise levels
expected from the diffraction patterns of such small samples, even when using
the spectacular brilliance of XFELs, and due to the challenge of assembling
several 2D diffraction patterns into a 3D diffraction volume when the
orientation of the sample is unknown and the noise is large. But several papers
have shown that the problem is tractable \cite{Veit2009Noise,NeTeDuaneLoh2009Reconstruction,Fung2008Structure}, all
what is missing is an experimental demonstration. The computational cost of
these approaches is extremely high so a strong investment in high
performance parallel software allied with a similar investment in hardware will
be crucial to bring these ideas to fruition.



%%% Local Variables: 
%%% mode: latex
%%% TeX-master: "Thesis"
%%% End: 
